\lecture{13}{October 07, 2020}{Example of First Isomorphism Theorem}

General idea of today's lecture: $1$\textsuperscript{st} iso thm and the conversation about quotient groups from last week are useful for studying a group G from the point of view of onto hom's $\phi: G \to$ something else.

\begin{exmp}
	$G' = \ZZ^2 = \{(a, b) : a, b \in \ZZ\}, (a, b) + (c, d) = (a+c, b+d)$. Note:  $\ZZ^2$ is an abelian group (vector addition is commutative).

	$G = F_2 = \text{``free group on $2$ generators''} = \{\text{finite strings using symbols }a, b, a^{-1}, b^{-1}, e\}$, with the operation of concatenation.

	Define $\phi: F_2 \to \ZZ^2$ by $\phi(a) = (1, 0), \phi(b) = (0, 1)$ and send anything else to where you would have to sent it to make $\phi$ a homomorphism, e.g., $g \in F_2, g = a^3 b^{-2} a^5$; then 
	\begin{align*}
		\phi(g) &= 3\phi(a) - 2 \phi(b) + 5\phi(a)\\
				&= (8, -2).
	\end{align*}

	The function $\phi$ is onto, since $\phi(a^c b^d) = (c, d)$, for every $(c, d) \in \ZZ^2$.

	Then, by the 1st iso thm, $\ZZ^2 \simeq F_2/\ker\phi$. (Which implies that, e.g., $F_2/\ker\phi$ is abelian.)

	So far, we have that $\ZZ^2 \simeq F_2/\ker\phi$. So, the idea is: learn something about $F_2$ by understanding what $\ker\phi$ is. Thus, we want to understand what $\ker\phi$ is.
	
	For example, $aba^{-1}b^{-1} \in \ker\phi$. More generally, if $g_1, g_2 \in F_2$, then  $g_1g_2g_1^{-1}g_2^{-1} \in \ker\phi$.

	So,  $\ker\phi \supset \text{group generated by all expressions of the form }g_1g_2g_1^{-1}g_2^{-1}$. 
	It is left to the reader to show that the equality holds.
	(Think about what it really means for something to be in the kernel.)

	Remember, $F_2$ is not abelian. And a group  $G$ is abelian $\iff gh = hg \iff ghg^{-1}h^{-1}$.

	The 1st isomorphism theorem says that the elements of $\ZZ^2$ are representing cosets of $\ker\phi$. This means that $g \in \ker\phi \iff \pi(g) = \text{itentity in } F_2/\ker\phi$. 

	All this to say: $\phi F_2 \to \ZZ^2$ collapses  $\ker\phi$ to a point and collapses each coset of $\ker\phi$ to a different point.

	Remember that $\ker\phi = \langle \text{commutators} \rangle$. The commutators are exactly the stuff that, if they're not identity, the group is not abelian.

	So, $F_2$ is not abelian because $ \langle \text{commutators} \rangle \neq \{\text{identity element}\}$ and we get an abelian group when we quotient out by this subgroup. 

	\vspace{1em}

	In summary:
	\begin{enumerate}[label = --]
		\item We characterized $\ker\phi$ as  $ \langle \text{commutators} = g_1g_2g_1^{-1}g_2^{-1} \rangle$ and commutators prevent abelian-ness. So, $F_2/\ker\phi$ should be abelian. And, 1st iso thm implies $F_2/\ker\phi \simeq \ZZ^2$, which is abelian.
	\end{enumerate}
\end{exmp}
