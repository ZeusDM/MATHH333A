\lecture{18}{October 19, 2020}{Ideals}

Our plan is to build up the analog of normal subgroups and quotient groups (kernels of homomorphisms), but for rings.

\begin{defn}
	An ideal in a ring $(R, +, \cdot)$ is a non-empty subset  $I \subseteq R$, such that
	\begin{enumerate} 
		\item $I$ is closed under $+$.
		\item  For all $r \in R$ and $s \in I$, then $rs \in I$.
	\end{enumerate}
\end{defn}

\begin{lem}
	If $\phi: R \to R'$ is a ring homomorphism, then $\ker\phi = \{r \in R \mid \phi(r) = 0_{R'}\}$ is an ideal of  $R$.
\end{lem}
\begin{dem}
	Left to the reader. (It is straightfoward).
\end{dem}

\begin{lem}
	$I$ is an ideal if, and only if $I \neq \varnothing$ and any linear combination $r_1s_1 + \cdots + r_ks_k$ is in  $I$.
\end{lem}

Prove this on your own.

\begin{exmp}
	\begin{enumerate}
		\item Kernels of ring homomorphisms.
		\item (Principal ideal generated by $a$) If $a \in R$, then $\{ra \mid r \in R\}$ forms an ideal. We write $(a)$ to refer to this ideal.
		\item More generally, given $a_1, \dots, a_n$, the ideal denoted by $(a_1, \dots, a_n)$ is $\{a_1r_1 + \cdots a_nr_n\}$ is an ideal.
	\end{enumerate}
\end{exmp}

\begin{defn}
	A proper ideal of $R$ is an ideal of $R$ that is not $\{O_R\}$ nor $R$.
\end{defn}

\begin{lem}
	A proper ideal is never a subring.
\end{lem}

\begin{exmp}
	Let $\phi: \RR[x] \to \RR$ given by $\phi(p(x)) = p(17)$.
	Then, by the lemma,  $\ker\phi$ is an ideal in $\RR[x]$.

	\begin{align*}
		\ker\phi &= \{p \in \RR[x] \mid p(17) = 0\}\\
				 &= \{f(x)(x-17) \mid f \in \RR[x]\}\\
				 &= (x - 17).
	\end{align*}
\end{exmp}

\begin{prop}\label{prop:l18}
	If $F$ is a field, then any ideal in $F[x]$ is principal.
\end{prop}

\begin{alg}[Polynomial division]
	Given $g \in R[x]$ and $f \in R[x]$ monic, there are unique polynomials $p, r \in R[x]$ such that $g = fp + r$, and  $\deg(r) < \deg(f)$.
\end{alg}

\begin{defn}
	If $R$ is a ring and every ideal of $R$ is principal, then it is called a principal ideal domain.
\end{defn}

\begin{prop}[Rewrite of \cref{prop:l18}]
	If $F$ is a field, then $F[x]$ is a principal ideal domain.
\end{prop}
\begin{dem}[of \cref{prop:l18}]
	Assume that $I \neq \{0\}$. Pick $f(x) \in I$ such that  $f(x) \neq 0$ and $\deg(f)$ is minimal. Say that $f(x) = a_nx^n + \cdots + a_0$.
	
	Since $F$ is a field, there is a multiplicative inverse of $a_n$ in $F$.
	Thus, let $\tilde f = a_n^{-1} \cdot f \in I$, and is monic; also with minimal degree.
	
	Since $\tilde f \in I$, we have $(\tilde f) \subseteq I$.

	Fix $g \in I$. Then, by polynomial division, there are unique $p, r \in F[x]$, with $\deg(r) < \deg(\tilde f)$ such that  $g = p\tilde f + r$. Then, $r \in I$, which implies  $r = 0$. Thus, $g = p\tilde f$.
	This means that  $I \subseteq (\tilde f)$.

	Therefore,  $I = (\tilde f)$.
\end{dem}

\begin{lem}
	If $R$ is a ring, then there exists only one ring homomorphism $\phi: \ZZ \to R$.
\end{lem}
\begin{dem}
	Define, $\phi(0) = 0_R$ for $n > 0$, $\phi(n) = 1_R + 1_R + \cdots + 1_R$, where $1_ R$ appears $n$ times, and $\phi(-n) = -\phi(n)$. This is the only ring homomorphism that exists.
\end{dem}

Note that $\ZZ$ is also a principal ideal domain.

So, let $\phi: \ZZ \to R$ be the unique ring homomorphism. Then $\ker\phi$ is an ideal in $\ZZ$, so it's principal.

\begin{defn}
	The characteristic of $R$ is the positive integer generating $\ker\phi$. If no such integer exists, then $\ker\phi = \{0\}$; in this case, we'll say that the characteristic is $0$.
\end{defn}
