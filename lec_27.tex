\lecture{27}{November 20, 2020}{Field Extensions}

\begin{dem}[of Eisenstein's Criterion]
	Suppose $f(x) = c_0 + \cdots + c_nx^n \in \ZZ[x]$ such that $p \nmid a_n$ and $p \mid a_0, a_1, \dots, a_{n-1}$. We shall prove that $f$ is reductible in $\QQ[x]$implies $p^2 \mid a_0$.

	Suppose $f = gh$, with $g, h \in \ZZ[x]$ and $\deg g, \deg h > 0$.
	Consider the ring homomorphism $\psi_p: \ZZ \to \ZZ/p\ZZ$ given by taking each coefficient modulo $p$. Then, 
	\[ \psi_p(g)\cdot\psi_p(h) = \psi_p(f) = a_nx^n.\]

	We claim that $\psi_p(g) = c_rx^r$ and $\psi_p(h) = c_sx^s$. 

	Suppose $\psi_p(g) = k_rx^r + \cdots + k_{r'}x^{r'}$ and $\psi_p(h) = c_sx^s + \cdots + c_{s'}x^{s'}$, with $k_r, k_{r'}, c_s, c_{s'} \neq 0$ and $r + s = n$. If $r' + s' < n$, then the $(r' + s')^\mathrm{th}$ coefficient of  $\psi_p(f)$ is $c_{s'}k_{r'}$, but it is also  $0$, which is an absurd. Therefore, $r' + s' = n \implies r = r'$ and $s = s'$.

	Back to $\ZZ[x]$,  $g$ and $h$ have constant coefficients multiples of $p$, therefore, the constant term of $f$ is multiple of $p^2$. 
\end{dem}

From now on, our focus will be constructing fields and studying their properties and knowing how to tell when a polynomial is irreductible will be extremely helpful.

\begin{defn}[Field Extension]
	If $K$ is a field and $F \subset K$ is a subfield, we say that $K$ is a \emph{field extension of $F$} and we write $K / F$.
\end{defn} 

\begin{exmp}
	A \emph{number field} is any subfield of $\CC$, so $\CC$ is a field extension of $F$. 

	If $F$ is a number field, then $\QQ \in F$, because $1 \in F \implies \ZZ \in F \implies \QQ \in F$.
\end{exmp}

\begin{defn}[Algebraic Numbers]
	Suppose $\alpha \in K$ and $K / F$. Then, $\alpha$ is \emph{algebraic over $F$} if there exists a monic polynomial $f \in F[x]$ such taht  $f(\alpha) = 0_K$.
	If $\alpha$ is not algebraic over $F$, then $\alpha$ is called \emph{transcendental over $F$}.
\end{defn}

\begin{exmp}
	$2\pi i \in \CC$ is algebraic over $\RR$, since it is a root of $f(x) = x^2 + 4\pi^2 \in \ZZ[x]$.
	However, $2\pi i \in \CC$ is transcendental over $\QQ$.
\end{exmp}

\begin{exmp}
	If $\alpha \in F$, $\alpha$ is algebraic over $F$.
\end{exmp}

\begin{lem}
	Given $\alpha \in K$, $K / F$. Then $\alpha$ is algebraic over $F$ if, and only if, $\phi_a: F[x] \to K$ is not one-to-one, where  $\phi_\alpha(p(x)) = p(\alpha)$.
\end{lem}

\begin{dem}
	$\phi_\alpha$ is not one-to-one $\iff \ker\phi_\alpha \neq \{0\} \iff$ there exists $F[x], f \neq 0$ and $\phi_\alpha = f(\alpha) = 0 \iff$ there exists a monic polynomial $\tilde f$ such that $\tilde f = 0$.
\end{dem}

So, suppose $\alpha \in K$ is algebraic over $F$. Then, $\ker\phi_\alpha \subset F[x]$ is not $0$. Recall that the kernel of a ring homomorphism is an ideal, so $\ker\phi_\alpha$ is a non-zero ideal in $F[x]$. Since $F[x]$ is a principal ideal domain,  $\ker\phi_\alpha = \langle f(x) \rangle$, for some $f \in F[x]$.

\begin{prop}\label{l27:prop4}
	Assume $\alpha \in K$ is algebraic over $F$. Then, for a given $f \in F[x]$, the following are equivalent:
	\begin{enumerate}
		\item $f$ is the monic polynomial of smallest degree in  $F[x]$ such that  $f(\alpha) = 0$.
		\item $f$ is irreducible in $F[x]$ and $f(\alpha) = 0$.
		\item $ \langle f(x) \rangle = \ker \alpha$ and $ \langle f(x) \rangle$ is maximal.
		\item $f(\alpha) = 0$ and if $g \in F[x]$ such that $g(\alpha)$, then $f \mid g$.
	\end{enumerate}
\end{prop}

\begin{defn}
	The polynomial that satisfies \cref{l27:prop4} is called the \emph{irreducible polynomial for $\alpha$ over $F$}. The degree of this polynomial is called the \emph{degree of $\alpha$ over $F$}.
\end{defn}
