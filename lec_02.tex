\lecture{2}{September 11, 2020}{Groups}

In the last class, we focused on binary (associative) operations.

\subsection{Defining Groups}

\begin{defn}[Notation]
	If $a, b \in S$, then $ab$ or $a \cdot b$ will commonly be used to denote $f(a,b)$. We will also commonly call this operation a \emph{product}.
\end{defn}

Associativity allows us to be less careful when writing down long products.

\begin{exmp}
	In general, $a_1a_2a_3a_4a_5a_6a_7$ has no meaning. However, if the binary opperation is associative, no matter in which order we do the product, there will be no ambiguity about what value the expression have.
\end{exmp}

\begin{defn}
	A binary operation on $S$ is called \emph{commutative} if for all $a, b \in S$, $ab = ba$ holds.
\end{defn}

\begin{exmp}~
	\begin{enumerate}
		\item $(\RR, +)$,  $(\CC, \cdot)$ have commutative binary operations.
		\item $(\mathcal{M}_n(\RR), \text{ matrix multiplication})$ has a non-commutative operation.
		\item $(\RR, \text{ distance})$, i.e, $f(a, b) = |a-b|$, has a commuative, but non-associative operation.
	\end{enumerate}
\end{exmp}

\begin{defn}
	Given $S$ equipped with a binary opperation, we say $(S, \cdot)$, has an identity element if there exists  $e \in S$ such that, for all $a \in S$, $a \cdot e = e \cdot a = a$ holds.
\end{defn}

\begin{exmp}~
	\begin{enumerate}
		\item $(\RR, +)$ has  $0$ as an identity.
		\item $(\RR, \cdot)$ has  $1$ as an identity.
		\item $(\mathcal{M}_n(\RR),\ \text{matrix multiplication})$ has $I_n$ as an identity.
	\end{enumerate}
\end{exmp}

\begin{defn}
	An element $a$ of  $(S, \cdot)$, that has an identity element (which we are going to call $e$), is called invertible if there exists $b \in S$ so that $ab = ba = e$.
\end{defn}

\begin{exmp}~
	\begin{enumerate}
		\item Every element of $(\RR, +)$ is invertible.
		\item Every element, except $0$, of $(\RR, \cdot)$ is invertible. 
		\item Some elements, but not all, of $\mathcal{M}_n(\RR)$, equipped with matrix multiplication, are invertible.
	\end{enumerate}
\end{exmp}

\begin{defn}
	A \emph{group} is a set $(G, \cdot)$ with a binary opperation so that:
	\begin{enumerate}
		\item The binary opperation is associative.
		\item There exists an identity element in $G$.
		\item Every element in $G$ is invertible.
	\end{enumerate}

	If $\cdot$ is commutative, $G$ is called an \emph{abelian group}.
\end{defn}

\begin{exmp}~
	\begin{enumerate}
		\item $(\RR, +)$ is a group.
		\item $(\CC, +)$ is a group.
		\item $(\ZZ, +)$ is a group.
		\item $(\RR\backslash\{0\}, \cdot)$ is a group.
		\item $(\CC\backslash\{0\}, \cdot)$ is a group.
		\item $(\ZZ\backslash\{0\}, \cdot)$ is not a group, because $2$ does not have an inverse element.
		\item[--] However, $(\QQ\backslash\{0\}, \cdot)$ is a group.
		\item $\mathcal{M}_n(\RR)$, equipped with matrix multiplication is not a group, because the zero matrix does not have an inverse element.
		\item[--] However, if we define $GL_n(\RR) = \{A \in \mathcal{M}_n(\RR)\ :\ \text{$A$ is invertible}\}$, then $GL_n(\RR)$, equipped with matrix multiplication is a group.\footnote{It is important to prove that matrix multiplication is closed under $GL_n(\RR)$. In addition, this is the first example of a non-abelian group.}
		
		\item Define $D_8 = \{\text{affine bijections}\ T:\RR^2 \to \RR^2\ \text{such that}\ T(\mathcal{S}) = \mathcal{S}\}$, where $\mathcal{S} = \{(0,0), (0,1), (1, 0), (1, 1)\}$, also known as the standard unit square.
			
	\end{enumerate}
\end{exmp}

