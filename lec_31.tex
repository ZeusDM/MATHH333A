\lecture{31}{December 07, 2020}{Corollaries of the Multiplicative Degree Theorem}

Let's study some corollaries of the ``multiplicative degree theorem'' from last class.

\begin{cor}
	Let $F \subset K$ be fields and $K / F$ a finite field extension, in which $[K : F] = n$.
	Then, for all $\alpha \in K$, $\alpha$ is algebraic over $F$ and $\deg_F(\alpha)$ divides $n$.
\end{cor}

\begin{dem}[Algebraic]
	If $\alpha$ was trancendental, then $F(\alpha)$ is a $\infty$-dimension vector space over $F$.
	
	However, since $F(\alpha)$ is a subspace of $K$, it must have a finite dimension, thus $\alpha$ is algebraic.
\end{dem}

\begin{dem}[Division]
	Using the multiplicative degree formula, we have that \[[K:F] = [K:F(\alpha)] [F(\alpha):F].\] Therefore, $\deg_F(\alpha) = [F(\alpha) : F]$ divides  $[K:F] = n$.
\end{dem}

\begin{cor}
	Let $F \subset K \subset L$ be fields, and $\alpha \in L$ is algebraic over $F$. Then, $\alpha$ is algebraic over $K$ and $\deg_K(\alpha) \leq \deg_F(\alpha)$,
\end{cor}

\begin{dem}[Algebraic]
	$\alpha$ is algebraic over $F$  $\implies$ there exists a polynomial $f \in F[x]$ with  $\alpha$ as a root $\implies$ there exists a polynomial $f \in K[x]$ (namely, the same polynomial) with $\alpha$ as a root $\implies$ $\alpha$ is algebraic over $K$.
\end{dem}

\begin{cor}
	Let $\alpha_1, \dots, \alpha_n$ be algebraic over a  field $F$. Then, $[F(\alpha_1, \dots, \alpha_n) : F] < \infty$.
\end{cor}

\begin{sk}
	Induction over $n$.
\end{sk}

\begin{cor}\label{l31:cor4}
	If $K/F$ is a finite extension, then there are $\alpha_1, \dots, \alpha_n$ such that $K = F(\alpha_1, \dots, \alpha_n)$.
\end{cor}

\begin{sk}
	Let $L_0 = F$.

	If $L_i \neq K$, then define $L_{i+1} = L_i(\alpha_{i+1})$, for some $\alpha_{i+1} \in K$ but not in $L_i$. 

	This process increases the dimension of $L_\bullet$, but the dimension of $L_\bullet$ may not be greater than the dimension of  $K$. Thus, the process has to terminate, i.e., $F(\alpha_1, \dots, \alpha_i) = L_i = K$.
\end{sk}


\begin{cor}
	If $K/F$, then the subset of algebraic elements over $F$ is a subfield of $K$.
\end{cor}

\begin{dem}
	Denote this subset by $\overline F$. $1, 0 \in \overline F$, since $x, x - 1 \in F[x]$. So $\overline F$ is a field if, and only if, given $\alpha, \beta \in \overline F$, the elements $\alpha + \beta, \alpha\beta, \alpha^{-1}$ are in $\overline F$.

	Note that $\alpha + \beta, \alpha\beta, \alpha^{-1} \in F(\alpha, \beta)$. Since $F(\alpha, \beta) / F$ is a finite field extension, any element of $F(\alpha, \beta)$ is algebraic over $F$, which implies that they are on $\overline F$.
\end{dem}

\begin{cor}[Cor. 15.3.8 on Artin]
	Let $K/F$ and $K'/F$ be field extensions. Let $L$ be the field generated by $K, K'$. If $[K:F]$ and $[K':F]$ are coprime, then \[[L:F] = [K:F] [K':F].\]
\end{cor}

\begin{sk}
	By the multiplicative degree theorem, we have that $[K:F]$ and $[K':F]$ divide $[L:F]$. Since $[K:F]$ and $[K':F]$ are coprime, this implies that $[K:F][K':F]$ divides $[L:F]$.

	It suffices to show that $[L:F] \le [K:F][K':F]$, i.e.,  $[L:K] \le [K':F]$.
\end{sk}
