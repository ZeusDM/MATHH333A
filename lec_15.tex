\lecture{15}{October 12, 2020}{Rings}

For groups, our prototype was the \emph{symmetries} of a geometric object.

For rings, our prototypes will be $(\ZZ, +, \cdot)$, and $(\{f : \RR \to \RR\}, + , \cdot)$.

\begin{defn}
	A ring is a set $R$, equipped with two binary opperations, $+$ and $\cdot$, satisfying:
	\begin{enumerate}
		\item $(R, +)$ is an abelian group.
		\item The operation $\cdot$ is commutative, associative, and has an identity.
		\item For all $a, b, c \in R$, it holds that $(a + b) \cdot c = a \cdot c + b \cdot c$.
	\end{enumerate}
\end{defn}

\begin{defn}[Notations]
	We denote $(R, +)$ as $R^+$. We denote the additive element as $0$. We denote the multiplicative element as $1 $.
\end{defn}

\begin{defn}
	A subring of a ring $R$ is a subset $S \subset R$, equipped with the same opperations as $R$. 
\end{defn}

Subrings of $\CC$ were studied as examples of rings in the lecture. The notes for this segment of the class are to do.
