\lecture{22}{October 28, 2020}{Creating fields from rings}

Given a ring $R$, how can we use it to create a fielf $F$?

Philosophically, there are two ways to do this:
\begin{enumerate}
	\item add elements to $R$ to create some field $F$ in which $R$ is a subring.
	\item kill elements of $R$, e.g., create $R/I$.
\end{enumerate}

\vspace{1em}

For ($i$), we can only do that if $R$ does not contain zero-divisors, i.e., if $R$ is an integral domain.

\begin{defn}[Field of Fractions]
	Let $R$ be an integral domain (a ring with no zero-divisors, i.e., $xy = 0$ implies $x = 0$ or $y = 0$).
	Given $R \times (R - \{0_R\})$, let $\sim$ be an equivalence relation defined by $(a, b) ~ (c, d) \iff ad = bc$.

	Then the field of fractions of $R$ is the set of equivalence classes of this relation. It has the following notions of addition and multiplication:
	\begin{enumerate}
		\item $(a, b) + (c, d) = (ad + bc, bd)$;
		\item $(a, b) (c, d) = (ac, bd)$.
	\end{enumerate}

	As always, whenever an object has different representations, it is important to show that the operations are consistent (if I change $(a, b)$ to an equivalent $(a', b')$, then the result is the same).
\end{defn}

\begin{prob}
	Prove that the set $R \times (R - \{0\})$, equipped with the operations defined above, is a ring.
\end{prob}

\begin{exmp}
	For $(\ZZ, +, \cdot)$,  $F(\ZZ) = \{[(a, b)] : a, b \in \ZZ\}$. In this case, we have $F(\ZZ) \approx \QQ$.
\end{exmp}

Note also that in $F(R)$, the additive identity is $[(0, 1)]$, and  $(0, 1) \sim (0, a)$ for $a \in R - \{0\}$; and the multiplicative identity is $[(1, 1)]$.

So, to show that  $F(R)$ is a field, it suffices to show that  $[(a, b)]$, where $a \neq 0$, has a multiplicative inverse. If $a \neq 0$, then $[(b, a)] \in F(R)$ and $[(a, b)] \times [(b, a)] = [(ab, ba)] = [(1, 1)]$.

Notationally, we will often denote $[(a, b)]$ as $\frac{a}{b}$.

 \begin{prop}
	 $F(R)$ is the ``smallest'' field containing $R$.
\end{prop}

\begin{thm}[Mapping principle]
	If $F_1$ is a field, $R$ is an integral domain and $R \subset F_1$, then there is a ring homomorphism $\phi: F(R) \to F_1$. Namely, $\phi$ maps $\frac{a}{b}$ to $ab^{-1}$.
\end{thm}

\begin{prop}
	The map $\psi: R \to F(R)$, defined by $r \mapsto [(r, 1)]$ is one-to-one.
\end{prop}

Before moving to ($ii$), let's introduce some terminology.
\begin{enumerate}[label = \textbullet]
	\item $R$ is a ring (not necessarily an integral domain).
	\item $u \in R$ is a \emph{unit} if it has a multiplicative inverse.
	\item $a$ \emph{divides} $b$, written as $a \mid b$, if there is $q \in R$ such that $b = aq$. 
	\item $a$ is a \emph{proper divisor} of  $b$ if $a$ is not a unit and there is $q \in R$, $q$ is not a unit and $b = aq$.
	\item  $a$ and $b$ are \emph{associates} if each divides each other.
	\item $a$ is \emph{irreductible} if $a$ does not have proper divisors.
	\item $a$ is \emph{prime} if it holds that $a \mid bc \implies a \mid b$ or  $a \mid c$.
\end{enumerate}

\begin{lem}
	\begin{enumerate}[label = \textbullet, before = \leavevmode\vspace{-\baselineskip}]
		\item $u$ is a unit if, and only if, $(u) = R$.
		\item $a \mid b \iff (b) \subset (a)$.
		\item $a$ is a proper divisor of $b$ if, and only if, $(b) \subsetneq (a) \subsetneq R$.
		\item $a$ and $b$ are associates if, and only if, $(a) = (b)$.
		\item $a$ is irreducible if, and only if, $(a) \subsetneq R$ and there is no $c \in R$ such that $(a) \subsetneq (c) \subsetneq R$.
		\item $a$ is prime if, and only if, $bc \in (a) \implies b \in (a)$ or $c \in (a)$.
	\end{enumerate}
\end
{lem}

\begin{defn}[Prime ideal]
	A prime ideal $I$ is an ideal such that, if $bc \in I$, then $b \in I$ or $c \in I$.
\end{defn}

\begin{defn}
	An ideal $I \neq R$ of a ring $R$ is called \emph{maximal} if, whenever $J$ is an ideal and $J \supset I$, then $J = I$ or $J = R$.
\end{defn}

In the next class, we will discuss the following theorem:
\begin{thm}
	$R / I$ is a field if, and only if,  $I$ is maximal.
\end{thm}

\begin{lem}\label{l22:fieldifftwoideals}
	A ring $R$ is a field if, and only if, $R$ contains exactly two ideals, namely $\{0\}$ and $R$.
\end{lem}
\begin{dem}
	If $R$ is a field, then if $J$ is an ideal, suppose $J \neq \{0_R\}$. Then there is some $a \in J$. However, since $R$ is a field, $a$ is a unit. Therefore, $(a) = R$, which implies that $J = R$.

	Conversely, for all $a \neq 0$, $(a)$ must be $R$, thus $a$ is a unit.
\end{dem}
