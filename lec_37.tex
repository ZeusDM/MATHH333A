\lecture{37}{December 29, 2020}{Supplementary Lecture IV}

\subsection{Galois Theory}

\begin{defn}[Isomorphism and Automorphism of Field Extensions over a Field]
	Let $K, K'$ be field extensions of $F$ and let $\rho: K \to K'$ is be an isomorphism. We call $\rho$ an \emph{isomorphism over $F$,} if $\rho(f) = f$, for all $f \in F$. Alternatively, we may call it a \emph{$F$-isomorphism}.

	If such isomorphism over $F$ exists, we say that $K$ and $K'$ are \emph{isomorphic over $F$}. Alternatively, we may say that $K$ and $K'$ are \emph{$F$-isomorphic}.

	If $K = K'$, we say that $\rho$ is an automorphism over $F$. Alternatively, we may call it a \emph{$F$-automorphism}.
\end{defn}

\begin{defn}[Galois Group]
	Given $K / F$ a field extension, let $\Aut_F(K)$ be the group of $F$-automorphisms of $K$, with the operation of composition. We call it the \emph{Galois group of $K$ over $F$}, denoted by $G(K/F)$.
\end{defn}

\begin{prop}
	The order of $G(K/F)$ divides $[K:F]$.
\end{prop}

\begin{defn}[Galois Extension]
	A finite extension $K/F$ is called a \emph{Galois extension} if, and only if,  \[
		|G(K/F)| = [K : F].
	\]
\end{defn}

\begin{exmp}
	$\CC/\RR$ is a Galois extension.
\end{exmp}

From now on, in this lecture, all fields have characteristic $0$.

\begin{lem}
	Let $K, K'$ be extensions of $F$. 
	Let $f \in F[x]$, and let $\sigma$ be an $F$-isomorphism from $K$ to $K'$.
	If $\alpha$ is a root of $f$ in $K'$, then $\sigma(\alpha)$ will be a root of $f$ in $K'$.
\end{lem}

\begin{lem}
	Let $K, K'$ be extensions of $F$.
	Suppose $K = F(\alpha_1, \dots, \alpha_n)$. Let $\sigma, \sigma'$ be $F$-isomorphisms from $K$ to $K'$.
	If $\sigma(\alpha_i) = \sigma'(\alpha_i)$ for all  $i$, then $\sigma = \sigma'$.
\end{lem}

\begin{cor}[$K = K'$] \label{l37:cor7}
	Let $K / F$ be a field extension.
	If a $F$-automorphism $\sigma$ fixes all $\alpha_i$'s, then $\sigma$ is the identity automorphism.
\end{cor}

\begin{lem}
	Let $f$ be a irreducible polynomial over $F$, and let $\alpha, \alpha'$ be roots of $f$ in $K, K'$, respectively. There exists a unique $F$-isomorphism $\sigma: F(\alpha) \to F(\alpha')$ sending $\alpha \to \alpha'$. If  $F(\alpha) = F(\alpha')$, then $\sigma \in G(F(\alpha)/F)$.
\end{lem}

\begin{prop}\label{l37:prop9}
	Let $f \in F[x]$. Then, $L/F$ contains at most one splitting field for $f$ over $F$.
\end{prop}

\begin{dem}
	If $L$ contains a splitting field of $f$, then $f$ splits completely in $L$, so $f = (x-\alpha_1)\dots(x-\alpha_n)$ in $L$.

	Thus, the only splitting field for $f$ in $L$ is $F(\alpha_1, \dots, \alpha_n)$.
\end{dem}

\begin{prop}
	Let $f \in F[x]$, then any two splitting fields of $f$ over $F$ are $F$-isomorphic.
\end{prop}

\begin{dem}
	Let $K_1, K_2$ be two splitting fields over $F$. $K_1$ is a extension of $F$ generated by finitely many algebraic elements, thus $[K_1 : F] < \infty$.

	By the \hyperlink{l34:thm8}{Primitive Element Theorem}, finite extensions of characteristic $0$ are generated by a single element, i.e., there exists $\gamma \in K$ such that $K_1 = F(\gamma)$.
	Since $K_1 = F(\gamma) = F[\gamma]$, the elements of $K_1$ are of the form $a_0 + a_1\gamma + \cdots + a_{(\deg\gamma - 1)}\gamma^{(\deg\gamma - 1)}$.

	Let $g(x)$ be the minimal polynomial for $\gamma$ over $F$. Since $g(x) \in F[x]$ and $K_2 / F$, there exists a field extension of $K_2$ so that $g$ has a root $\gamma'$ in $L$. 

	Consider the map $\phi: K_1 \to F(\gamma')$, defined by \[
		a_0 + a_1\gamma + \cdots + a_{(\deg\gamma - 1)}\gamma^{(\deg\gamma - 1)} \mapsto 
		a_0 + a_1(\gamma') + \cdots + a_{(\deg\gamma - 1)}(\gamma')^{(\deg\gamma - 1)}.
	\]

	This map is a $F$-isomorphism that sends $\gamma$ to $\gamma'$. Therefore, $F(\gamma')$ is isomorphic to $K_1$  $\implies$ $F(\gamma')$ is a splitting field for $f$.

	Finally, $F(\gamma')$ and $K_2$ are splitting fields for $f$ in $L$, thus \cref{l37:prop9} implies that $F(\gamma') = K_2$, which finishes the proof.
\end{dem}

\begin{defn}[Fixed Fields]
	Let $K$ be a field and $H \le \Aut(K)$. The \emph{fixed field} of $H$ is \[
		K^H = \{ \alpha \in K : \sigma(\alpha) = \alpha,\ \forall \sigma \in H\}.
	\]
\end{defn}

\begin{prop}
	$K^H$ is a subfield of $K$.
\end{prop}

\begin{thm}[Important Theorem about Fixed Fields]\label{l37:thm13}
	Let $H$ be a finite subgroup of $\Aut(K)$. Suppose $H$ is a finite group. Let $\beta_1 \in K$ and let $\{\beta_1, \dots, \beta_r\}$ be the $H$-orbit of $\beta_1$. Then
	\begin{enumerate}[label = (\alph*)]
		\item the minimal polynomial for $\beta_1$ over $K^H$ is $g(x) = (x-\beta_1)\cdots(x-\beta_n)$.
		\item $\beta_1$ is algebraic over $K^H$, and $\deg_{K^H}\beta_1 = r = \text{size of $H$-orbit of $\beta_1$}$. So, $\deg_{K^H}\beta_1$ divides $|H|$.
	\end{enumerate}
\end{thm}
	
\begin{rem}
	$\sigma \in H$ if, and only if, $\sigma \in G(K/K^H)$. Thus, $H = G(K/K^H)$.
\end{rem}

\begin{dem}
	Let $g(x) := (x - \beta_1)\cdots(x - \beta_n) \in K[x]$. The coefficents of $g(x)$ are $s_i(\beta_1, \dots, \beta_n)$. Since the elements of $H$ permute the orbit $\{\beta_1, \dots, \beta_n\}$, they fix the coefficients of $g$; thus, the coefficents of $g(x)$ are in $K^H$, i.e., $g(x) \in K^H[H]$.

	Let $h(x) \in K^H[x]$ such that $h(\beta_1) = 0$. Pick $\sigma \in H$ such that $\sigma(\beta_1) = \beta_i$. Then, $0 = \sigma(0) = \sigma(h(\beta_1)) = h(\sigma(\beta_1)) = h(\beta_i)$; thus all  $\beta_1, \dots, \beta_n$ are roots of $h(x)$, which implies that $g(x) \mid h(x)$. Therefore, $g(x)$ is the minimal polynomial for $\beta_1$ over $K^H$.

	This implies that $\beta_1$ is algebraic and $\deg_{K_H}\beta_1 = \deg g = r$.
\end{dem} 

\begin{lem}\label{l37:lem14}
	Let $K$ be an algebraic extension of $F$ such that $K$ is not a finite extension. Then, there are elements in $K$ whose degrees over $F$ are arbitrarily large.
\end{lem}

\begin{thm}[Fixed Field Theorem]
	Let $H$ be a finite subgroup of $\Aut(K)$. Then, $[K : K^H] < \infty$ and $[K : K^H] = |H|$.
\end{thm}

\begin{dem}
	By \cref{l37:thm13}, $\deg_{K^H}\beta$ is limited, for  $\beta \in K$. Thus, by \cref{l37:lem14},  $K$ has to be a finite extension of $K^H$, i.e., $[K:K^H] < \infty$. By the \hyperlink{l34:thm8}{Primitive Element Theorem}, there exists $\gamma \in K$ such that $K = K^H(\gamma)$. By \cref{l37:cor7}, if $\sigma(\gamma) = \gamma$, then $\sigma$ is the identity. Therefore, the only element in $H$ that fixes $\gamma$ is the identity; which implies that the $H$-orbit of $\gamma$ has size $|H|$.

With not great effort, we can imply that $[K^H(\gamma) : K^H] = |H|$.
\end{dem}

\begin{thm}[Main Theorem of Galois Theory]
	Let $K$ be a Galois extension of $F$. Then, the correspondance between the subgroups of $G(K/F)$ and the intermediate fields of $K$ over $F$ defined by $H \mapsto K^H$ and $G(K/L) \mapsfrom L$ is a bijection.
\end{thm}


