\lecture{3}{September 14, 2020}{Subgroups}

Let's look more closely to $D_8 = \{\text{affine bijections}\ T : \RR \to \RR\ \text{such that}\ T(\mathcal{S}) = \mathcal{S}\}$.
 
\begin{prop}\label{l3:d8}
	$D_8$ is a group. The order of the group $D_8$ is $8$.
\end{prop}

\begin{prop}
	Let $r, h \in D_8$ be described as follows:
	\begin{enumerate}
		\item $r$ denotes the rotation of $\mathcal{S}$ by $90^\circ$, counter-clockwise.
		\item $h$ denotes the reflect across the horizontal perpendicular bisector. 
	\end{enumerate}

	\begin{center}
		\def\svgwidth{5cm}
		\import{./figures/}{d8_rotation_reflection.pdf_tex}
	\end{center}
	
	If $\phi \in D_8$, then $\phi$ can be expressed as  $\phi = \phi_{n} \circ \phi_{n-1} \circ \cdots \phi_1$, where  $\phi_i = h$ or $\phi_i = r$, for all $i$. 
\end{prop}

The proposition above should resemble the concept of basis in Linear Algebra. In some sense, $h$ and $r$ gererate the group $D_8$.

\begin{exmp}
	Let $d$ be the reflection through the diagonal line through $(0,0)$ and $(1,1)$. We have $d = h \circ r \circ r \circ r = hr^3$.
	\begin{center}
		\def\svgwidth{4.5cm}
		\import{./figures/}{d8_diagonal.pdf_tex}
	\end{center}
\end{exmp}

\begin{exmp}
	Let $v$ be the reflection though the vertical perpendicular bisector. We have $v = hr^2$.
\end{exmp}

Note that $h^2 = r^4 = e$, and $2 \cdot 4 = 8$, which is the number of elements in $D_8$. What a coincidence, isn't it?

\begin{defn}
	A \emph{subgroup} $H$ of a group $(G, \cdot)$ is a subset of  $G$ that is a group itself, with respect to the same operation $\cdot$.
\end{defn}

\begin{exmp}~
	\begin{enumerate}
		\item If $G$ is a group, it has an identity, say $e$. Then $\{e\}$ is a subgroup of $G$. 
		\item $G$ is always a subgroup of $G$.
	\end{enumerate}
\end{exmp}

\begin{lem}
	Given a a group $G$, a non-empty subset $H \subset G$ is a subgroup of $G$ if, and only if, both following conditions are met:
	 \begin{enumerate}
		 \item $ab \in H$, for all  $a, b \in H$.
		 \item $a^{-1} \in H$, for all  $a \in H$.
	\end{enumerate}
\end{lem}

\begin{exmp}
	$2\ZZ = \{\text{even integers}\}$ is a subgroup of $(\ZZ, +)$.
\end{exmp}

\begin{defn}[Symmetric group on $n$ elements]
	Given $n \in N$, define $S_n = \{\text{bijections}\ \tau: \{1, 2, \dots, n\} \to \{1, 2, \dots, n\}\}$, equipped with composition.
\end{defn}

\begin{exmp}
	Let $n = 5$, then consider $\tau:
	\begin{pmatrix}
		1 & 2 & 3 & 4 & 5\\
		3 & 4 & 1 & 2 & 5
	\end{pmatrix}.$ Then $\tau \in S_5$.

	Alternatively, we can use the following notation for $\tau = (13)(24)(5)$, which is called \emph{cycle notation}.
\end{exmp}

\begin{exmp}
	Consider $\tau':
	\begin{pmatrix}
		1 & 2 & 3 & 4 & 5\\
		2 & 4 & 5 & 1 & 3\\
	\end{pmatrix}$. We can write $\tau' = (1 2 4)(3 5)$, using cycle notation.
\end{exmp}

\begin{exmp}
	Consider $\tau'':
	\begin{pmatrix}
		1 & 2 & 3 & 4 & 5\\
		2 & 3 & 4 & 5 & 1\\
	\end{pmatrix}$. We can write $\tau'' = (1 2 3 4 5)$, using cycle notation.
\end{exmp}

\begin{rem}
	Cycle notation is ``not unique'', e.g., $(1 2 3 4 5) = (3 4 5 1 2)$.
\end{rem}
