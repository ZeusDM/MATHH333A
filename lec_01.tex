\lecture{1}{September 09, 2020}{Binary Operations}

\subsection{Why Algebra?}

Algebra is the study of symmetry. An object has a symmetry when we can do something to it (transform it in some way) and without changing its appearance.

\begin{exmp}
	A circle has a rotational symmetry: if we rotate the circle about its center, we get the same circle.
\end{exmp}

\begin{exmp}
	The algebraic equation $x^2 + y^2 + z^2 - 3xyz = 0$ has a symmetry: for example, we can change the roles of $x$ and $z$, which gives us the same equation.
\end{exmp}

Symmetry appears all over Mathematics, so Algebra is a prevalent topic abroad Mathematics.

\subsection{Places where Algebra arises in Mathematics}

\paragraph{Number Theory.}
The following theorem will be proven in this course.

\begin{thm}[Fermat's Little Theorem]
	Let $p$ be a prime integer number. Let $a$ be a positive integer number. Then, $a^p - a$ is a multiple of $p$.
\end{thm}

\paragraph{Topology.}

\begin{thm}
	There is no continous bijection $f: S \to T$.

	\begin{center}
		\def\svgwidth{5cm}
		\import{./figures/}{topology_example.pdf_tex}
	\end{center}
\end{thm}

\begin{sk}
	Associate a ``group'' to $S$ and another to $T$. A continous bujection would send the $S$-group perfectly to the $T$-group. But the two groups are different.
\end{sk}

\subsection{Binary Operations}

\begin{defn}
	If $S$ is a set, then a \emph{binary operation} on $S$ is a function $f: S \times S \to S$. Here,  $S \times S = \{(a, b)\ |\ a, b \in S\}$.
\end{defn}

\begin{exmp}
	If $S = \RR$, then  $f(a, b) = a+b$ and  $g(a, b) = a \cdot b$ are binary operations.
\end{exmp}

\begin{exmp}
	If $S = \NN$, then $h(a, b) = a - b$ is not a binary operation.
\end{exmp}

\begin{defn}
	A binary operation $f: S \times S \to S$ is \emph{associative} if, for all $a, b, c \in S$,  \[ f(f(a, b), c) = f(a, f(b, c)). \] 
\end{defn}

\begin{exmp}
	If $S = \mathcal{M}_n(\RR)$, then $f(A, B) = AB$ is an associative binary operation.
\end{exmp}

\begin{exmp}
	If $S = \RR$, then $f(a, b) = a - b$ is a non-associative binary operation.
\end{exmp}

A key concept in Algebra is \emph{transformation}.

\begin{exmp}
	Let $S$ be a non-empty section. Define $g(S) = \{T : S \to S\}$. Then, composition is an associative binary operation on  $g(S)$, i.e., $f(T_1, T_2) = T_1 \circ T_2$ is an associative binary operation on $g(S)$.
\end{exmp}

