\lecture{23}{November 11, 2020}{Review lecture: Maximal Ideals}

Let's recall some definitions.

\begin{defn}[Ring]
	A set $R$, equipped with two operations $+$ and $\cdot$, is a ring if the following conditions are satisfied:
	\begin{enumerate}
		\item $(R, +)$ is an abelian group.
		\item The operaion  $\cdot$ is commutative, associative and has an identity.
		\item For all $a, b, c \in R$, it holds that $(a + b) \cdot c = a \cdot c + b \cdot c$.
	\end{enumerate}
\end{defn}

\begin{exmp}[Rings]
	$(\ZZ, +, \cdot)$ and $R[x]$ for any ring $R$ are examples of rings.
\end{exmp}

\begin{defn}[Field]
	A ring $R$ is a field if every non-zero element is a unit, i.e., if every non-zero element has a multiplicative inverse.
\end{defn}

\begin{exmp}[Fields]
	$(\RR, +, \cdot)$,  $(\CC, +, \cdot)$, $(\QQ, +, \cdot)$ and $(\ZZ/p\ZZ, +, \cdot)$ are examples of fields.
\end{exmp}

\begin{defn}[Ideal in a Ring]
	A non-empty subset $I \subset R$ is an ideal in $R$ if the following conditions are satisfied:
	\begin{enumerate}
		\item $I$ is closed under $+$.
		\item For all $r \in R$ and $s \in I$, it holds $rs \in I$.
	\end{enumerate}
\end{defn}

\begin{exmp}[Ideal]
	The set of the multiples of $7$ is an ideal in $(\ZZ, +, \cdot)$.
\end{exmp}

\begin{defn}[Maximal Ideal]
	An ideal $I$ in $R$ is a maximal ideal in $R$ if:
	\begin{enumerate}
		\item $I \neq R$
		\item There is no ideal $I'$ in $R$ such that $I \subsetneq I' \subsetneq R$.
	\end{enumerate}
\end{defn}

\begin{exmp}[Maximal Ideals]
	The maximal ideals of $\ZZ$ are the principal ideals generated by prime numbers, $ \langle p \rangle$.
\end{exmp}

The following two propositions will help us to understand how to transform a ring $R$ into a field by removing some of its elements.

\begin{prop}\label{l23:prop5}
	Let $\phi: R \to R'$ be a surjective ring homomorphism and let $I = \ker \phi$. \emph{(Remember: the $\ker\phi$ of a ring homomorphism is always an ideal.)} 

	$R'$ is a field if, and only if, $I$ is a maximal ideal.
\end{prop}
\begin{dem}
	From \cref{l22:fieldifftwoideals}, a ring is a field if, and only if, it contains precisely two ideals, itself and ${0}$.
	
	The correspondance theorem says that, for any onto ring homomorphism $\phi: R \to R'$, there exists a bijective correspondence between $\{\text{ideals in $R$ containing $\ker\phi$}\} \leftrightarrow \{\text{ideals in $R'$}\}$.
	This bijective correspondance is given by $I \mapsto \phi(I)$ and $\phi^{-1}(I) \mapsfrom I$.

	So, if $I = \ker\phi$ is a maximal ideal, all ideals in $R$ containing $I$ are $I$ and $R$; which, by the correspondance theorem, implies that $R'$ contains precisely two ideals $\implies$ $R'$ is a ring.

	Conversely, if $R'$ is a field, it contains precisely two ideals. Using the correspondance theorem, we have that there are only two ideals in $R$ that contain $I$. Since $I$ and $R$ are ideals that contain $I$, they are all of the ideals containing $I$. This means that there can't be an $I'$ such that $I \subsetneq I' \subsetneq R$, which implies that $I$ is a maximal ideal.
\end{dem}
\begin{cor}\label{l23:cor6}
	$I$ is maximal if, and only if, $R/I$ is a field.
\end{cor}
\begin{dem}
	Consider the natural ring homomorphism $\pi: R \to R/I$, given by $r \mapsto r + I$. The kernel of $\pi$ is $I$.

	Therefore, by \cref{l23:prop5}, we have that $R/I$ is a field if, and only if, $I$ is maximal.
\end{dem}
\begin{cor}
	The zero ideal $\{0_R\}$ of  $R$ is maximal if, and only if, $R$ is a field.
\end{cor}
\begin{dem}
	Plug in $I = \{0_R\}$ in \cref{l23:cor6}, and we're done.
\end{dem}

\begin{exmp}[Transforming $\ZZ$ into a field by killing elements]
	Since $ \langle p \rangle = p\ZZ$ is a maximal ideal of $\ZZ$, we conclude that $\ZZ/p\ZZ$ must be a field.
\end{exmp}
