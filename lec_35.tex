\lecture{35}{December 25, 2020}{Supplementary Lecture II}

\begin{lem}\label{l35:lem1}
	Let $F$ be a field of characteristic $0$, and let $K = F(\alpha, \beta)$. Then, for all but finitely many $c \in F$, $\gamma := \alpha+ c \beta$ is primitive, i.e., $K = F(\gamma)$.
\end{lem}
\begin{dem}[of \cref{l35:lem1}]
	Let $f(x)$ and $g(x)$ be the irreducible polynomials for  $\alpha$, $\beta$ over $F$, respectively. There exists a field extension $L$ in which $f$ and $g$ split completely. Let $\alpha_1 = \alpha, \alpha_2, \dots, \alpha_m \in K$ be the roots of $f$ and $\beta_1 = \beta, \beta_2, \dots, \beta_n \in K$ be the roots of $g$.

	Since the characteristic is zero, all $\alpha_i$'s and $\beta_j$'s are distinct.  Let $c \in F$, and let $\gamma_{ij} = \alpha_i + c \beta_j$. There are $mn$ of such $\gamma$.  Suppose $(i, j) \neq (k, \ell)$. Note that
	\begin{align*}
		\gamma_{ij} = \gamma_{k\ell} &\iff \alpha_i + c\beta_j = \alpha_k + c\beta_\ell \\
									 &\iff (\beta_j - \beta_\ell)c = \alpha_k - \alpha_i,
	\end{align*}
	which holds for at most one choice of $c$ in $F$.

	Thus, pick some $c \in F$ which is not any of the finitely many elements found above. Then, we have $\gamma_{ij} \neq \gamma_{k\ell}$ for all  $(i, j) \neq (k, \ell)$.  We shall prove that $\gamma := \gamma_{11} = \alpha_1 + c\beta_1 = \alpha + c\beta$ is primitive. Let $J = F(\gamma)$. Since $\gamma \in K$, we have $J \subset K$.
	Define a new polynomial, $h(x)$, by \[
		h(x) = f(\gamma - cx),
	\]
	which is a polynomial over $F(\gamma)$.

	We have shown previously that $\gcd(f(x), h(x))$ is the same when computed in $J[x]$ or in $K[x]$. In $K[x]$, $f(x) = (x-\beta)(x-\beta_2)\cdots(x-\beta_m)$. Since $\beta$ is a root of $h(x)$, but no $\beta_i, i > 1,$ is a root of $h(x)$, we have \[
		\gcd(f(x), h(x)) = (x-\beta).
	\]

	Since $f(x), h(x) \in J[x]$, we have that $x - \beta \in J[x] \implies \beta \in J \implies \alpha in J \implies K \subset J \implies K = J$.
\end{dem}

\begin{rem}
	Since $F$ has characteristic $0$, it cannot be a finite field. Thus, there is some $c$ such that \[F(\alpha + c\beta) = F(\alpha, \beta).\]
\end{rem}

\begin{dem}[of \cref{l34:pet}, the Primitive Element Theorem]
	%Recall \cref{l33:lem1}.

	Since $K/F$ is finite, $K$ is genereated by a finite set, i.e., there exists a finite basis $(\alpha_1, \dots, \alpha_n)$ for $K$ over $F$.

	Given $\gamma \in K$, there are $a_1, a_2, \dots, a_n$ such that $\gamma = a_1\alpha_1 + \cdots + a_n\alpha_n$.

	Thus,  \[K = F(\alpha_1, \dots, \alpha_n).\]

	Let's induct on $n$.

	\begin{enumerate}[label = \textbullet]
		\item 
		If $n = 1$, then $K = F(\alpha_1)$, so we're done!

		\item 
			If $n = 2$, then $K = F(\alpha_1, \alpha_2)$. By \cref{l35:lem1}, there exists some $\gamma$ such that $F(\alpha_1, \alpha_2) = F(\gamma)$, so we're done.

		\item 
		If $n \ge 3$, then we can use the induction hypothesis on the field $K' = F(\alpha_1, \dots, \alpha_{n-1})$ and say that  $K'$ is generated by a single element. Thus, $K' = F(\beta)$, for some $\beta \in F$. Thus, $K = F(\beta, \alpha_n)$, which is generated by a single element using the base case $n = 2$.
	\end{enumerate}
\end{dem}

\section{Splitting fields}

\begin{defn}[Splitting field]
	Let $F$ be a field and $f(x) \in F[x]$, not necessarily irreducible over $F$.
	There exists some field extension of $F$ such that $f(x)$ splits completely over $K$, i.e., \[
		f(x) = (x - \alpha_1) \cdots (x - \alpha_n).
	\]

	The \emph{splitting field of $f(x)$ over $F$} is the field $K = F(\alpha_1, \dots, \alpha_n)$.

	More vaguely, the splitting field of $f(x)$ over $F$ is the smallest (up to isomorphism) field extension of $F$ such that $f(x)$ splits completely.
\end{defn}

\begin{prop}
	For every $\beta \in K$, there exists a polynomial $p \in F[x_1, \dots, x_n]$ such that $p(\alpha_1, \dots, \alpha_n) = \beta$.
\end{prop}
