\lecture{5}{September 18, 2020}{Cyclic Groups}

Recall that every subgroup $S$ of $(\ZZ, +)$ is of the form $d\ZZ$, for some integer $d$.

Also, if  $a, b$ are integers, we can consider $S = a\ZZ + b\ZZ$, which is a subgroup of $\ZZ$. Therefore, $ S\ZZ = d\ZZ$. for some integer $d$.

Since $a, b \in S = a\ZZ + b\ZZ$, then $a, b \in d\ZZ$, which means that $d$ is a divisor of both $a, b$. 

Now, let $n \in \ZZ$ such that  $n$ divides both  $a$ and $b$. Thus, $n$ divides any number of the form $sa + rb$.
But,  $d\ZZ = a\ZZ + b\ZZ$, which means $d = ra + bs$, for right choices of $r$ and $s$. Therefore, $n$ divides $d$.

\begin{defn}
	 For $a, b \in \ZZ$, we define $d$ as above as the \emph{greatest common divisor} of  $a$ and $b$, which we denote by $\gcd(a, b)$.
\end{defn}

We have shown not only that $d$ is the greatest common divisor of $a$ and $b$, but also that any other common divisor of $a$ and $b$ divides $d$.

\begin{alg}[Euclidean Algorithm]
	
\end{alg}

\begin{exmp}
	Let $a = 314$ and  $b = 136$. We divide $314$ by $136$ and get $314 = 2\cdot 136 + 42.$ Thus, 
	\begin{align*}	
		n \in 314 \ZZ + 136\ZZ & \iff n = r \cdot 314 + s\cdot 136\\
							   & \iff n = r \cdot (2\cdot136 + 402) + s\cdot136\\
							   & \iff n = r \cdot (2r + s) \cdot 136 + r \cdot 42\\
							   & \iff n \in 136 \ZZ + 42\ZZ.
	\end{align*}

	Therefore, $\gcd(314, 136) = \gcd(136, 42)$.
	We can further use 
\end{exmp}

\begin{defn}
	Given $a, b \in \ZZ$,  $a, b \neq 0$, then  $a$ and $b$ are relatively prime if, and only if, $\gcd(a, b) = 1$.
\end{defn}

\begin{prop}
	The $\gcd(a, b)$ is the product of the prime powers common to prime factorizations of $a$ and $b$.
\end{prop}

\begin{exmp}
	Let $a = 52 = 2^2 \cdot 13$, and $b = 2^3 \cdot 3$. Therefore,  $\gcd(52, 24) = 2^2$.
\end{exmp}

\begin{cor}
	If $a$ and $b$ are relatively prime if, and only if, there are integers $r$ and $s$ such that $ra + sb = 1$.
\end{cor}

\begin{cor}
	Suppose $p$ is a prime. Then, given $a, b \in \ZZ$, if $p$ divides $ab$, therefore $p$ divides $a$ or $p$ divides $b$.  
\end{cor}

\begin{dem}
	If $p$ divides $a$, we are done.

	Suppose that $p$ does not divide $a$. Thus, $\gcd(p, a) = 1$. It implies that \[1 = rp + sa,\] for some integers $r$ and $s$. If we multiply both sides by b, we have \[b = rbp + sab.\]

	Notice that $p$ divides both $rbp$ and $sab$, therefore, $p$ divides their sum, which is $b$.
\end{dem}

\begin{thm}
	Let $G = (G, \cdot)$ be a group, let $I$ be a set, and let $\{H_i\}_{i \in I}$ be a family of subgroups of $G$ indexed by $I$. Then, the set \[\bigcap_{i \in I} H_i\] is a group.
\end{thm}

\begin{dem}
	We want to show:
	\begin{enumerate}[label = (\roman*)]
		\item $\bigcap_{i \in I}H_i \neq \varnothing$.

			For this item, $e \in \bigcap_{i = I}H_i$.

		\item $a, b \in \bigcap_{i \in I} H_i \implies ab \in \bigcap_{i \in I}$.

			For this item, $a, b \in H_i$, for all $i \in I$, which implies $ab \in H_i$ for all $i$
	\end{enumerate}
\end{dem}

Back to $(\ZZ, +)$. Given  $a, b \in \ZZ$, let $S = a\ZZ \cap b\ZZ$. By the last theorem,  $S$ is a subgroup. By Wednesday's theorem, $S = a\ZZ + b\ZZ = m\ZZ$, for some  $m \in \ZZ$. Since $m \in m\ZZ = a\ZZ \cap b\ZZ$, $m$ is a multiple of $a$ and $b$.

Now, for any number $n$ that is multiple of both $a$ and $b$ $\implies$ $n \in a\ZZ \cap b\ZZ = m\ZZ$  $\implies$ $n$ is a multuple of $m$.

\begin{defn}
	The $m$ described above is called the \emph{lowest common multiple} of $a$ and $b$, denoted by $\lcm(a, b)$.
\end{defn}

We have proved above only only that $m$ is the lowest common multiple, but also that $m$ divides every common multiple of $a$ and $b$.

\begin{defn}
	Let $(G,\cdot)$ be a group and  $x \in G$. Then the cyclic subgroup generated by $x$, denoted by $<x>$, is all powers of $x$, i.e., \[\left<x\right> = \{\dots, x^{-1}, e, x^1, x^2, \dots\}.\]
\end{defn}

\begin{thm}
	In $G$, let $\Gamma(x) = \{H \subseteq G : H\ \text{ is a subgroup of }\ G\ \text{and}\ x \in H\}$. Then \[\bigcap_{H \in \Gamma(x)} H = \left<x\right>.\]
\end{thm}
